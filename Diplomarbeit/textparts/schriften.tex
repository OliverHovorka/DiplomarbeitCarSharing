\documentclass[a4paper, 12pt]{report}

\usepackage{german}						%Deutsche Silbentrennung etc.
\usepackage[latin1]{inputenc}	%Umlaute in .tex Files normal schreibbar
\usepackage{a4}								%A4 Randeinstellungen
\usepackage{setspace}					%Gr��erer Zeilenabstand
\usepackage{courier}
\usepackage{times, mathptm}

\parindent=0pt		%Kein Einr�cken der ersten Zeile eines Absatzes
\parskip=12pt			%12pt Abstand zwischen 2 Abs�tzen
\doublespacing 		%Doppelter Zeilenabstand

\setlength{\headheight}{15pt}		%Kopfzeile vergr��ern (wegen 12pt Schriftgr��e)	
\addtolength{\textwidth}{1.5cm}	%Rechten Rand verkleinern

\begin{document}
	\rmfamily
	
	\chapter{�bersicht - rmfamily}
	
		\section{Der Java Community Process (JCP)}
	
			Seit 1998 existiert eine Unterabteilung von Sun Microsystems welche sich mit der Weiterentwicklung der Java Technologie besch�ftigte. 
			
			Diese Abteilung wurde in den Jahren 2000 und 2002 in Zusammenarbeit mit gro�en Industriefirmen �ffentlich zug�nglich gemacht und als so genannter Java Community Process eingerichtet welcher von einem Executive Committee geleitet wird.
	
	\sffamily
	
	\chapter{�bersicht - sffamily}
	
		\section{Der Java Community Process (JCP)}
	
			Seit 1998 existiert eine Unterabteilung von Sun Microsystems welche sich mit der Weiterentwicklung der Java Technologie besch�ftigte. 
			
			Diese Abteilung wurde in den Jahren 2000 und 2002 in Zusammenarbeit mit gro�en Industriefirmen �ffentlich zug�nglich gemacht und als so genannter Java Community Process eingerichtet welcher von einem Executive Committee geleitet wird.
				
	\ttfamily
		
	\chapter{�bersicht - ttfamily}

		\section{Der Java Community Process (JCP)}
	
			Seit 1998 existiert eine Unterabteilung von Sun Microsystems welche sich mit der Weiterentwicklung der Java Technologie besch�ftigte. 
			
			Diese Abteilung wurde in den Jahren 2000 und 2002 in Zusammenarbeit mit gro�en Industriefirmen �ffentlich zug�nglich gemacht und als so genannter Java Community Process eingerichtet welcher von einem Executive Committee geleitet wird.

\end{document}