\begin{flushleft}
	
	\subsection*{Zusammenfassung}

	\subsubsection*{Optimierung der Fahrzeugr�ckstellung f�r ein Car-Sharing-Unternehmen:}
	
	Ziel des Projektes jSAS ist die Portierung der bestehenden, auf Servlets basierenden, SASII Weboberfl�che zu einer benutzerfreundlichen Anwendung mit ergonomischer Oberfl�che.
	
	Um eine Performancesteigerung zur Vorversion zu erreichen und das Rechtemanagement besser auf die Anforderungen abzustimmen, wird anstatt einer PostgreSQL\tm Datenbank die neueste Version der SAP\tm Datenbank benutzt. Weiters soll die Rechtevergabe dynamisch und durch das Programm administrierbar sein.
	
	Die Verteilung des Programms und auch das automatische Einspielen von Updates und Patches soll mit Java WebStart\tm realisiert sein.
	
	\subsubsection*{�bersicht �ber die Diplomarbeit:}
	
	Diese Diplomarbeit stellt die Neuerungen der Version 1.5 der Java Entwicklungsumgebung sowie der Java Laufzeitumgebung vor. Besondere Ber�cksichtigung finden die Einf�hrung von generischen Konstrukten und Verbesserungen in der Syntax. 
	
	Die Ergebnisse werden speziell auf ihre Einsetzbarkeit im Projekt JSAS untersucht.
\end{flushleft}